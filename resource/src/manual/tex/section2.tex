\section{ماژول \ماژول‌پلین{}}
\label{بخش:ماژول}

این بخش به توضیحات مربوط به طراحی و ساختار
ماژول اختصاص داده شده است. هرچند اطلاع از این
موارد در به کاربردن و توسعهٔ ماژول توسط خواننده
مفید است، اما برای استفاده از آن نیازی به
دانستن این جزئیات نیست. لذا در صورتی که قصد
خواننده صرفاً فراگیری نحوهٔ کاربرد ماژول است
توصیه می‌شود به بخش~\رجوع{بخش:کاربرد} مراجعه شود.

\subsection{طرح انتزاعی}
در طراحی ماژول سعی شده با تعریف ساختاری انتزاعی
که در فرآیندهای پرداخت الکترونیکی متداول است
بستری قابل توسعه فراهم شود تا علاوه بر سرویس واسط
زرین‌پال، توسعه‌دهندگان بتوانند آن را برای 
استفاده از دیگر سرویس‌های مشابه نیز توسعه دهند.
لذا لازم است تعریف مجردی از اجزای دخیل در
فرآیند پرداخت الکترونیکی در اختیار داشته باشیم.

با توجه به رویکرد گفته شده، اجزای انتزاعی زیر
برای فرآیند پرداخت الکترونیکی در نظر گرفته شدند:
\begin{itemize}
	\فقره \متن‌ایتالیک{مدیر پرداخت} یا \مدیرپرداخت{}
	\فقره \متن‌ایتالیک{جلسهٔ پرداخت} یا \جلسهٔ‌پرداخت{}
	\فقره \متن‌ایتالیک{پرداخت} یا \پرداخت{}
\end{itemize}


